\documentclass[12pt]{article}
	\usepackage{amsmath}
	\usepackage{amsfonts}
	\usepackage{fullpage}
	\usepackage{slantsc}
	\usepackage{bold-extra}
	\usepackage{url}

\begin{document}
	\begin{center}
		{\Large \bf CSci 5608 Spring 2018}
	\end{center}
\medskip

{\bf Written Assignment}
\hfill
{\bf Xingzhe Xin, xinxx073@umn.edu, 5197296}
\hrule
\bigskip

\begin{center}
{\bfseries LEVEL OF DETAIL ALGORITHMS}
\end{center}

Games have always fascinated me, and the graphics in games have become closer to reality than ever
before. Most recently, Ureal Engine has been able to capture a person's whole body and facial expressions
and render 3D animation with the same expressions in real time.\footnote{Andy Serkis - Unreal 
Engine \url{https://www.youtube.com/watch?v=TxErDzsIdKI}} As impressive as this may be, the video
was rendered in real time by 4 GTX TITAN V graphics cards, each costing around \$3000. Before TITAN V level computing power
arrives in everyone's ultrabooks, we must make use of the limited hardware resouces we have. \par

Level of detail algorithms exist for this purpose. If we are watching a movie at a theater, the framerate is usually a fixed 24 fps
in order to produce a stable, steady experience to the audience. However, we soon discover that 24 fps does not work well in games.
When a movie is being filmed, the camera is either not moving at all, or moving at a very steady rate while pointing at some fixed position.
However, within a game, the main camera is constantly changing direction --  especially in First Person Shooters, where the camera
is constantly pointing left and right aiming for the enemies. Assuming that the monitor can display images at high frame rates,
he gaming experience increases dramatically when increasing the fps from 30 to 60, 120, and finally 240 fps, above which, the
increase is close to unnoticeable. With this discovery, games have been tring to maximize the amount of fps by lowering the level
of detail each image produces, and this is when LOD algorithms comes in handy. \par

The least suitable technique in this case might be image-driven-simplification due to its slow nature. \par
With vertex clustering, we could keep the total number of verticies in the scene below a certain amount. Since when the number
of verticies is stable, the amount of triangles in the scene might also be close to some constant value. With this method, video 
memory useage should also be relatively stable, since it stores the texture and vertex information of the scene. For different 
machines, we just need to change this "optimal reference number", which is basically just the number of cells, 
to adapt to the machine. The number could simply be calculated with a hardware benchmark program that executes when the game 
runs for the very first time. Vertex clustering also seems to be easy to implement. Comparing this to floating-cell clustering,
vertex clustering should operate faster, since it does not need to sort the verticies by their importance. Although it might not look
as good as using floating-cell clustering, games are, more often than not, performance driven, and vertex clustering might be the 
better option if we want to keep the frame rate steady. This algorithm is simplistic at its core, so it can easily be used in
different situations, across platforms that have varying performance.\par

Vertex decimation better preserves the shape of the object by keeping track of the topology, but it is hard to guarantee an outcome.
When we specify the distance to edge and distance to plane, we can't directly see how much impact we are making to the outcome. Those
criteria are also difference when the scene changes, thus the implementation should be harder. To stay at a constant frame rate, 
the parameters passed in to the algorithm would need constant adjustment. \par

With the two clustering techniques, we could control the average triangle size by limiting the size of the cell we are using. This 
seems quite a difficult task for vertex decimation, since the poly count after decimation is hard to determine. \par

And finally, with the quadric error metric algorithm, 

\begin{center}
	{\bf COHERENCE}
\end{center}
\end{document}
